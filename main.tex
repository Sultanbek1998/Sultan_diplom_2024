% ОБЯЗАТЕЛЬНО ИМЕННО ТАКОЙ documentclass!
% (Основной кегль = 14pt, поэтому необходим extsizes)
% Формат, разумеется, А4
% article потому что стандарт не подразумевает разделов
% Глава = section, Параграф = subsection
% (понятия "глава" и "параграф" из документа, описывающего диплом)
\documentclass[a4paper,14pt]{extarticle}

% Подключаем главный пакет со всем необходимым
\usepackage{diploma}

\addbibresource{literature.bib} %Import the bibliography file

% Пакеты по желанию (самые распространенные)
% Таблицы
\usepackage{longtable}
\usepackage{makecell}
% Картинки (можно встявлять даже pdf)
\usepackage[pdftex]{graphicx}

\usepackage{amsthm,amssymb, amsmath}
\usepackage{textcomp}

% Подсветка кода (все стили в файле)
\input{code_highlight.tex}

\begin{document}

% Титульник в файле titlepage.tex
\input{titlepage.tex}

% Содержание
\tableofcontents
\pagebreak

% ============================================
% ВВЕДЕНИЕ
% ============================================
\specialsection{Введение}

В данной дипломной работе проведено исследование моделирования протекания тока через одиночные квантовые точки различной формы и размера. Область исследования включает в себя изучение особенностей электронных структур и их влияние на электронные свойства квантовых точек.

Практическая значимость исследования состоит в возможности создания новых устройств на основе квантовых точек. На основе полученных результатов могут быть разработаны приборы с улучшенными электронными характеристиками, такие как высокоскоростные и низкопотребляющие транзисторы, светодиоды с контролируемыми оптическими свойствами, а также квантовые компьютеры и квантовые криптографические устройства.

Проблема, решаемая в данной работе, заключается в анализе энергетических уровней и волновых функций электронов в квантовых точках различной формы. Это позволяет лучше понять физические механизмы, определяющие электронные свойства квантовых точек, и предложить методы их оптимизации для конкретных приложений.

Для достижения поставленных целей были использованы методы математического моделирования и численного анализа. В частности, для расчета энергетических уровней и волновых функций были применены алгоритмы, реализованные в программных средах R-studio и Matlab.

Кроме того, в работе был проведен обширный анализ литературных источников, включая современные научные публикации и статьи, посвященные проблематике квантовых точек и их электронных свойств. Полученные результаты сопоставлены с предыдущими исследованиями, что позволяет уточнить и дополнить существующие знания в данной области.

\specialsection{Цель и задачи}
\label{Tasks}

\textbf{Цель:} Цель не должна совпадать с темой работы. Цель должна быть достижима (должен быть конечный результат) и проверяема. Исследование --- это процесс, и целью быть не может.

\textbf{Задачи}
\begin{enumerate}
    \item Расчёт положения уровней энергии в КТ и плотности состояний по крайней мере для кубических (кубоид) и сферических квантовых точек
    \item Уровней энергии $E$ для кубических квантовых точек 
    \item Уровней энергии $E$ для сферических квантовых точек
    \item Волновых функций $\psi$ для кубических квантовых точек 
    \item Волновых функций $\psi$ для сферических квантовых точек
\end{enumerate}

Достаточно задач. Обзор литературы наверное в задачи включать не будем. Лучше написать конкретно, что мы делаем (разработка алгоритма, программная реализация, расчёт конкретных параметров при определённых условиях и т.д.)

% ============================================
% ГЛАВА 1
% ============================================
\pagebreak
\section{Обзор литературы}

\subsection{Основное понятие о квантовых точках}

\paragraph{Что такое квантовая точка?}

Квантовая точка представляет собой наноструктурный объект, обладающий размерами порядка нанометра и обычно ограниченный в трех пространственных направлениях. Она может быть выполнена из полупроводникового материала и иметь свойства, сильно отличающиеся от их объемных аналогов из-за квантовых эффектов, таких как квантовая конфайнмент. Примеры таких эффектов включают увеличение энергетического разнообразия и изменение оптических и электронных свойств. Квантовые точки широко используются в электронике, фотонике, квантовых вычислениях и других областях науки и техники.

За последние три десятилетия был сделан значительный прогресс в технологии, позволяющий производить полупроводниковые структуры размером в нанометрах. Это масштаб длины, где действуют законы квантовой механики, и проявляется целый ряд новых физических эффектов. С одной стороны, можно проверять фундаментальные законы физики, а с другой стороны быстро возникают многочисленные потенциальные применения.\cite{vukmirovic}

Последнее слово в наноструктуре, где носители заряда ограничены во всех трех пространственных измерениях, называется квантовой точкой. За последние 15 лет квантовые точки были получены несколькими различными способами в широком диапазоне полупроводниковых материальных систем. Свойства квантовых точек и их возможные применения в значительной степени зависят от метода их получения, что может быть использовано в качестве критерия для классификации различных типов квантовых точек:\cite{vukmirovic}

\paragraph{Электростатические квантовые точки.} Квантовые точки можно создавать, ограничивая двумерный электронный газ в полупроводниковой гетероструктуре боковыми электростатическими воротками или вертикально с помощью техник травления [1, 2]. Свойства этого типа квантовых точек, иногда называемых электростатическими квантовыми точками, можно контролировать путем изменения приложенного потенциала к вороткам, выбора геометрии воротков или внешнего магнитного поля. Типичный размер этих точек составляет порядка 100 нм.\cite{vukmirovic}

\paragraph{Самоорганизующиеся квантовые точки.} Самоорганизующиеся квантовые точки получают в гетероэпитаксиальных системах с различными постоянными решетки. Во время роста слоя одного материала поверх другого происходит образование наномасштабных островков [3], если толщина слоя (так называемый промежуточный слой) больше определенной критической толщины. Этот режим роста называется режимом Странски-Крастанова. Самыми распространенными экспериментальными техниками роста эпитаксиальных наноструктур являются молекулярно-лучевая эпитаксия (MBE) и химическое осаждение металлоорганических соединений из газовой фазы (MOCVD) . Поскольку материал квантовых точек внедрен в другой материал, мы также будем называть их внедренными квантовыми точками. Самоорганизующиеся квантовые точки обычно имеют боковые размеры порядка 15–30 нм и высоту порядка 3–7 нм.\cite{vukmirovic}

\paragraph{Коллоидные квантовые точки.} Очень разным подходом к получению квантовых точек является синтез одиночных кристаллов размером всего в несколько нанометров с использованием химических методов. Такие точки, полученные таким образом, называются нанокристаллами или коллоидными квантовыми точками. \cite{vukmirovic} Их размер и форма могут быть контролируемы длительностью, температурой и молекулами лиганда, используемыми при синтезе. Коллоидные квантовые точки обычно имеют сферическую форму. Они часто меньше внедренных квантовых точек, а диаметр иногда составляет всего 2–4 нм.

Квантовые точки позволили изучить множество фундаментальных физических эффектов. Электростатические квантовые точки можно контролируемо заряжать с желаемым количеством электронов и, таким образом, создавать целую периодическую систему искусственных атомов, что предоставляет множество данных, из которых можно получить дополнительное понимание многократной физики фермионных систем. Были исследованы одноэлектронный транспорт и эффект блокировки Кулона, а также режим Кондо-физики. Одним из наиболее захватывающих аспектов исследования квантовых точек, безусловно, является перспектива использования состояния точки (спиновое состояние, экситон или заряженный экситон) в качестве кубита в квантовой информационной обработке. Было достигнуто согласованное управление состоянием экситона в одиночной точке, выбранной из ансамбля самоорганизующихся квантовых точек, а также манипулирование спиновым состоянием в электростатических квантовых точках. Теоретический и экспериментальный прогресс в области явлений, связанных со спином, в квантовых точках был рассмотрен в обзоре. Эти результаты кажутся многообещающими, хотя управление большим количеством квантовых точек кубитов еще не осуществимо, в основном из-за сложности управления взаимодействиями кубит-кубит.

Практические применения квантовых точек, конечно, не отстают от этих захватывающих областей фундаментальной науки с квантовыми точками. Например, коллоидные квантовые точки нашли несколько актуальных приложений, таких как флуоресцентные биологические метки, высокоэффективные фотоэлектрические солнечные элементы и светодиоды на основе нанокристаллов. Самоорганизующиеся квантовые точки находят основное применение как оптоэлектронные устройства: лазеры, оптические усилители, источники одиночных фотонов и фотодетекторы. В этом обзоре будет сосредоточено на теоретичес

\subsubsection{Методы одиночных частиц}

Хотя квантовые точки кажутся маленькими и простыми объектами, при более детальном рассмотрении их структуры с атомистической точки зрения становится очевидной их высокая сложность. Учитывая, что решеточные константы базовых полупроводниковых материалов обычно находятся в пределах порядка \( 0.5 \) нм, можно оценить, что одна самоорганизованная квантовая точка содержит приблизительно \( 10^6 \) ядер и еще большее количество электронов, взаимодействующих друг с другом с использованием кулоновских сил на большие расстояния. Даже самые маленькие коллоидные квантовые точки содержат тысячи атомов. Это явно показывает, что прямое решение гамильтониана многих частиц квантовых точек не является практичным подходом и требует разработки умных и эффективных методов. В этом разделе рассматриваются методы, которые сводят проблему к эффективному уравнению для одной частицы.

Более двадцати лет назад Брус предложил простой метод эффективной массы для расчета ионизационных энергий, электроаффинностей и оптических переходных энергий в полупроводниковых нанокристаллах. В рамках модели Бруса энергии одиночных частиц (электронов или дырок) \( E \) и волновые функции \( \psi(r) \) удовлетворяют уравнению Шрёдингера, заданному как:

\begin{equation}
    \left[ -\frac{1}{2m^*} \nabla^2 + P(r) \right] \psi(r) = E \psi(r)
    \end{equation}
    
где \( m^* \) - это эффективная масса электрона или дырки. В уравнении 1 используется система атомных единиц, в которой уменьшенная постоянная Планка \( \bar{h} \), масса электрона \( m_0 \) и заряд электрона \( e \) все равны 1, и это будет использоваться в дальнейшем. Для простоты уравнение 1 предполагает, что частица должна быть заперта внутри точки, то есть потенциал за пределами точки бесконечен. Это упрощающее предположение легко устранить, добавив более реалистичный конфинирующий потенциал \( V_{\text{conf}}(r) \).

\( P(r) \) в уравнении 1 - это дополнительный потенциал, вызванный наличием поверхности квантовой точки. Он имеет определенную аналогию с электростатическими изображениями потенциалов в случае, когда заряд находится недалеко от поверхности металла или интерфейса между двумя диэлектриками. Его можно получить, вычислив энергию взаимодействия между голым электроном и его индуцированным потенциалом экранирования. Дополнительная энергия взаимодействия электрона в точке r внутри квантовой точки по сравнению соответствующим значением в объеме затем равна \( P(r) \).

Для моделирования двухчастичных возбуждений (например, электрон + дырка = экситон) Брус ввел электростатическую энергию взаимодействия между этими частицами как

\begin{eqnarray}
    V(r_1, r_2) &=& \pm \frac{e^2}{\epsilon |r_1 - r_2|} \nonumber \\
    && \pm PM(r_1, r_2) + P(r_1) + P(r_2)
    \end{eqnarray}
    
где \( \epsilon \) - это диэлектрическая проницаемость, \( P_M \) соответствует взаимодействию заряда одной частицы с индуцированным поляризационным потенциалом поверхности другой частицы, в то время как термины P описывают взаимодействие заряда одной частицы с ее собственным индуцированным поляризационным потенциалом поверхности, как это было описано ранее. Плюс (минус) знак предназначен для двух частиц с одинаковым (противоположным) зарядом. Эффективный гамильтониан экситона тогда задается как

\begin{eqnarray}
    H_{\text{exciton}} &=& -\frac{1}{2m_e} \nabla^2 e - \frac{1}{2m_h} \nabla^2 h \nonumber \\
    && - \frac{e^2}{\epsilon |r_e - r_h|} - PM(r_e, r_h) + P(r_e) + P(r_h)
    \end{eqnarray}
    
Решение собственной задачи этого гамильтониана можно записать аналитически как:

\begin{equation}
    E^* \approx E_g + \frac{\pi^2}{2R^2} \left( \frac{1}{m_e} + \frac{1}{m_h} \right) - \frac{\alpha ce^2}{\epsilon R} + \text{small term}
    \end{equation}

где \( \alpha_c = 2 - \frac{Si(2\pi)}{\pi} + \frac{Si(4\pi)}{2\pi} \approx 1.8 \), а \( Si(x) \) - это функция синуса интеграла \( Si(x) = \int_{0}^{x} \frac{\sin t}{t} dt \). Последний член в уравнении 4 происходит из последних трех членов в уравнении 3. Следует отметить, что \( P(r) = \frac{PM(r, r)}{2} \), поэтому \( PM(re, rh) \) и \( P(re) + P(rh) \) точно сокращаются, когда \( re = rh \) и приводят к небольшому члену, когда \( re \) и \( rh \) не равны. Этот маленький член часто можно игнорировать на практике для сферических квантовых точек.

Отмена поляризационных членов дает нам руководство для общего подхода к расчету экситонов в нанокристаллах. На первом этапе рассчитываются энергии одночастичных состояний из уравнения 1 без поляризационного члена. На втором этапе энергии одночастичных состояний дополнительно учитывается экранированное взаимодействие между электроном и дыркой. Однако следует иметь в виду, что такой подход является приближением, основанным на классических электростатических соображениях. Он игнорирует эффекты, такие как динамическое экранирование и локальные поляризационные эффекты диэлектрической функции. Полученные таким образом состояния одночастичных частиц не являются квазичастицами из обычной формализации GW (Собственные энергии уравнения 1 с членом P - это энергии квазичастиц, соответствующие электронной аффинности и потенциалу ионизации.). Однако такие состояния одночастичных частиц являются естественным продолжением состояний одночастичных частиц, рассматриваемых в других наноструктурах, таких как квантовые ямы и сверхрешетки. Они также полностью соответствуют собственным состояниям, определенным в теории функционала плотности, описанной ниже. Остальная часть раздела 2 будет посвящена теоретическим основам и методологиям расчета этих состояний.

\subsubsection{Теория функционала плотности}

В рамках теории функционала плотности (DFT) , многотелесная задача гамильтониана сводится к набору уравнений для одночастичных уровней Кона-Шэма , которые записываются в виде:

\begin{equation}
    \left( -\frac{1}{2} \nabla^2 + V_{\text{ion}} + V_{\text{H}} + V_{\text{XC}} \right) \psi_i(r) = \varepsilon_i \psi_i(r)
    \end{equation}
    
В уравнении выше $\psi_i(r)$ и $\varepsilon_i$ обозначают волновые функции и энергии орбиталей Кона-Шэма, $V_{\text{ion}}(r)$ - потенциал всех ядер в системе, а $V_H(r)$ - потенциал Хартри от электронов, заданный как

\begin{equation}
    V_{\text{H}}(r) = \int \frac{Z \rho(r')}{|r - r'|} \, dr'
    \end{equation}

где

\begin{equation}
    \rho(r) = \sum_{i} |\psi_i(r)|^2
    \end{equation}

электронная плотность заряда системы. Суммирование в уравнении (7) происходит по всем заполненным орбиталям Кона-Шама. Потенциал обмена-корреляции $V_{\text{XC}}$ в уравнении (5) должен учитывать все остальные эффекты взаимодействия электрон-электрон за пределами простого отталкивания Кулона (описанного в $V_H$). Точная форма этого потенциала неизвестна и должна быть аппроксимирована. Самое широко используемое приближение - это локальное приближение плотности (LDA), где предполагается, что $V_{\text{XC}}$ зависит только от локальной электронной плотности заряда и принимает такое же значение, как и в свободном электронном газе этой плотности. Уравнения 5 и 7 должны быть решены самосогласованно до достижения сходимости.

Расчеты на основе теории функционала плотности по-прежнему требуют значительных вычислительных ресурсов, в частности из-за необходимости проведения самосогласованных расчетов. Также необходимо вычислить все орбитали $\psi_i$ на каждой итерации, тогда как в полупроводниковых системах часто интересуются лишь несколькими состояниями в области щели, определяющими оптические и транспортные свойства системы.

\paragraph{Какие свойства обладают квантовые точки?}

Квантовые точки обладают рядом уникальных свойств, включая квантовый размерный эффект, высокую светоизлучающую способность, широкий спектр энергетических уровней, и возможность контролировать их свойства путем изменения их размеров и формы. Эти свойства делают квантовые точки перспективными материалами для применений в области фотоники, оптоволоконной связи, и квантовых вычислений.

\paragraph{Каким образом создаются квантовые точки?}

Квантовые точки обычно создаются путем использования различных методов, таких как методы химического осаждения из раствора, эпитаксиальные методы роста на подложках, и методы самоорганизации. Эти методы позволяют создавать квантовые точки с различными размерами и формами, что определяет их свойства и потенциальные применения.

\paragraph{Какова роль квантовых точек в современной науке и технике?}

Квантовые точки играют важную роль в современной науке и технике, поскольку они являются ключевыми элементами для разработки новых технологий в области электроники, фотоники, квантовых вычислений и биомедицинских применений. Они имеют потенциал для создания новых видов сенсоров, оптических устройств, источников света и многих других устройств, которые могут привести к новым достижениям в различных областях науки и техники.

\subsection{Обзор методов моделирования транспорта через квантовые точки}

Физика полупроводниковых квантовых точек является одним из ключевых направлений для исследований в будущем развитии электроники. Транспортные и оптические явления в этих наноструктурах могут привести в будущем к инновационным применениям на основе их уникальных свойств. Уже было продемонстрировано, что многие материалы можно манипулировать для производства квантовых точек. В частности, были получены кремниевые квантовые точки, открывая новые горизонты для этого хорошо известного полупроводника. Для описания этого явления желательно разработать теоретические основы, позволяющие быстро предсказывать его свойства как в равновесии, так и под воздействием внешнего потенциала. Достигнув этого, станет возможным проводить будущие исследования по физике систем с огромным количеством атомов, таких как молекулы квантовых точек или кристаллы квантовых точек с разумной вычислительной стоимостью.\cite{bolivar}

Десятилетиями подход с использованием эффективной массы (EMA) применялся для описания физических явлений в объемных и структурах квантовых ям. Однако для изучения низкоразмерных систем, таких как квантовые провода и точки, исследователи проверили его применимость, сравнив результаты, полученные этой теорией, с результатами, полученными более сложными методами, такими как псевдопотенциальные и теснозаполненные подходы (TB). Таким образом, многие исследования показали, что EMA может точно описывать наноструктуры, превзойдя начальные ожидания. Главным преимуществом EMA является его способность выполнять простые и быстрые вычислительные процедуры, что делает его более эффективным, чем другие более сложные техники. Тем не менее, эти другие методы предоставляют отличные инструменты для улучшения результатов, полученных с использованием EMA, и также необходимы для определения предела применимости этого подхода.\cite{bolivar}

Для этой цели Нике и др. провели интересное сравнение между параметризацией TB структуры зоны Si и методами EMA, приходя к заключению, что последние становятся менее точными при увеличении ограничения. В случае квантовых точек полученные результаты показали, что при эффективном диаметре около 8,5 нм метод EMA недооценивает основное состояние дырок на более чем 25\% (эффективный диаметр определяется как диаметр сферы с тем же объемом, что и исследуемая наноструктура). В случае электронов наблюдается тенденция к завышению основного состояния, что также приводит к завышению оптического зазора квантовой точки. Авторы в своей работе утверждают, что одной из причин этого избыточного ограничения в методе EMA может быть невозможность распространения волновой функции за пределы точки из-за используемого потенциала для ее моделирования, тогда как результаты TB показывают, что волновая функция состояний может частично деликализовываться по атомам H, пассивирующим поверхность точки. В результате объем, занимаемый волновой функцией в TB, может быть больше, чем в EMA, последний избыточно ограничивает электроны и дырки.\cite{bolivar}

Продолжая эту линию рассуждений, мы предлагаем в этой работе найти поправку к размеру кубических и сферических квантовых точек при расчете спектров квантовых точек с помощью метода k·p. Для этого мы искали размер точки, необходимый для расчета k·p, обеспечивающий такую же энергию основного состояния, как и в TB для конкретной точки. Следуя наблюдениям Нике и др., для получения такого же основного состояния расчеты k·p потребуют более крупных точек, чем TB. Установив эту связь, должно быть возможно найти аналитическое выражение, связывающее эквивалентные размеры для k·p и TB, и использовать его для коррекции результатов k·p. Теория будет проверена путем сравнения спектров квантовой точки после коррекции размера. Таким образом, мы сможем проверить надежность подхода.

\paragraph{Теоретический формализм}

Мы начинаем с определения прямоугольного блока с длинами \( L_x \), \( L_y \) и \( L_z \) в каждом направлении, который используется в качестве области огибающих функций (Рис. 1). Мы предлагаем расширить электронные огибающие функции следующим образом:

\begin{figure}[htbp]
    \centering
    \includegraphics[width=15cm]{images/1-fig.bolivar.png}
    \caption{\label{fig:bolivar1} Схематическое представление области огибающих функций.Серый куб - это область пространства, где волновые функции не обращаются в нуль. Исследуемая система моделируется внутри этой области. На рисунке мы изображаем сферический квантовый точечный дефект.\cite{bolivar}.}
\end{figure}

\begin{eqnarray}
    F(\vec{r}) &=& \sqrt{\frac{8}{L_x L_y L_z} }\sum_{r=1}^{N_x} \sum_{s=1}^{N_y} \sum_{t=1}^{N_z} a_{r,s,t} \sin(k_rx) \sin(k_sy) \sin(k_tz),
    \end{eqnarray}

где \( a_{r,s,t} \) - коэффициенты разложения, \( k_r = \frac{r\pi}{L_x} \), \( k_s = \frac{s\pi}{L_y} \), \( k_t = \frac{t\pi}{L_z} \), \( N_x \), \( N_y \) и \( N_z \) - количество значений, используемых для каждого \( k \), а \( \frac{8}{L_x L_y L_z} \) - нормализационная константа. Учитывая симметрию кубических и сферических квантовых точечных дефектов (которые изучаются в данной работе), мы упростили вышеуказанное общее разложение, используя \( N_x = N_y = N_z = N \).

\begin{figure}[htbp]
    \centering
    \includegraphics[width=15cm]{images/2-fig.bolivar.png}
    \caption{\label{fig:bolivar2} Унименсиональная схема потенциала, используемая для моделирования области.\cite{bolivar}.}
\end{figure}

Квантовые точечные дефекты моделируются внутри домена с помощью следующего потенциала:

Представленный метод основан на ряде Фурье и предполагает, что за пределами области та же проблема повторяется периодически с периодичностью $L_x$, $L_y$ и $L_z$ в каждом направлении. Тем не менее, основное улучшение по сравнению с нормализованным методом расширения плоских волн заключается в том, что границы области представляют собой бесконечные барьеры типа дельта-функции Дирака (см. рис. 2).

\begin{eqnarray}
    V(\vec{r}) = \begin{cases}
    0, & \text{если } \mathbf{r} \text{ находится внутри точки}, \\
    V_{\text{ext}}, & \text{если } \mathbf{r} \text{ находится вне точки}.
    \end{cases}
    \end{eqnarray}
    
Таким образом, можно изучать квантовую точечную дефектность любой формы.

\subsubsection*{Плотностно-функциональная теория для модельной квантовой точки:Выход за рамки локального приближения плотности}

Теория функционала плотности (DFT) является эффективным инструментом для определения электронной структуры твердых тел. Хотя изначально она была разработана для непрерывных систем с кулоновским взаимодействием, DFT также была применена к решетчатым моделям, таким как модель Хаббарда или модели без спина. Эти решетчатые модели часто позволяют получить точные решения - либо аналитически, либо на основе численных методов, - которые могут служить ориентирами для оценки качества приближений.\cite{schenk}

Очень популярным в приложениях твердого тела является локальная аппроксимация плотности (LDA), в которой энергия обмена-корреляции неоднородной системы строится с использованием локальной аппроксимации относительно однородной электронной системы. Недавно была предложена решеточная версия LDA для одномерных систем, где базовая однородная система может быть решена с использованием анзаца Бете. Например, было продемонстрировано, что анзац Бете в LDA хорошо описывает низкочастотные, длинноволновые возбуждения взаимодействующей одномерной системы, то есть жидкость Луттингера.\cite{schenk}

\paragraph{Статическая теория функционала плотности (DFT)}

Решеточная версия DFT опирается на тот факт, что существует однозначное соответствие между локальными потенциалами $f_{\text{виг}}$ и математическим ожиданием основных состояний занятости узлов $f_{\text{ниг}}$. Поэтому в принципе возможно выразить все величины, которые могут быть получены из волновой функции основного состояния, как функции плотностей. Занятости узлов как функция потенциалов могут быть найдены из производных энергии основного состояния по отношению к локальному потенциалу.\cite{schenk}

\begin{equation}
    n_i = \frac{\partial E_0}{\partial v_i}
    \end{equation}

Для определения потенциалов из плотностей удобно определить функцию

\begin{equation}
    F(\{n_{\text{i}}\}) = \min_{\Psi\rightarrow\{n_{i}\}} \left\{ \langle\Psi|\hat{T} +\hat{V}|\Psi\rangle\right\}
\end{equation}

где $f_{\text{ниг}}$ указывает на то, что минимизация ограничена такими волновыми функциями, которые дадут заданные занятости узлов $f_{\text{ниг}}$. Здесь $T$ и $V$ - кинетическая и взаимодействующая части гамильтониана соответственно. Энергия основного состояния получается путем минимизации функции

\begin{equation}
    E(\{n_{\text{i}}\}) = F(\{n_{\text{i}}\}) + \sum_{i} v_i n_i
    \end{equation}

по отношению к $n_i$. Когда мы минимизируем $E$ при условии постоянного числа частиц, мы получаем потенциал с точностью до добавочной константы (множителя Лагранжа):

\begin{equation}
    v_i = -\frac{\partial F}{\partial n_i} + \lambda
\end{equation}

Важным шагом к практической реализации DFT является использование невзаимодействующего вспомогательного гамильтониана $\hat{H}_s$ (гамильтониан Кона-Шэма), чтобы рассчитать профиль плотности

\begin{equation}
    \hat{H}^{s} = \hat{T} + \sum_i v^{s}_{i} \hat{n}_i
\end{equation}

где потенциалы $v_s$ должны быть выбраны таким образом, чтобы в основном состоянии $H_s$ занятости узлов $n_i$ были такими же, как и во взаимодействующей модели. Аналогично взаимодействующей системе, энергия основного состояния системы Кона-Шэма находится путем минимизации

\begin{equation}
    E^{s}(\{n_{\text{i}}\})= F^{s}(\{n_{\text{i}}\}) + \sum_i v^{s}_{i} \cdot n_{i}
\end{equation}

Комбинируя (12) и (15), получаем

\begin{eqnarray}
    E(\{n_{\text{i}}\}) & = & E^{s}(\{n_{\text{i}}\}) + E^{\text{HXC}}(\{n_{\text{i}}\}) + \nonumber \\
    & & \sum_i (v_i - v^{s}_{i}) n_i;
\end{eqnarray}

с энергией Хартри-обменно-корреляционной, определенной как

\begin{eqnarray}
    E^{HXC}(\{n_{\text{i}}\}) & = & F(\{n_{\text{i}}\}) - F^{s}(\{n_{\text{i}}\});
\end{eqnarray}

Условие, при котором как $E$, так и $E_s$ минимальны для одного и того же набора занятости узлов $n_i$, требует, чтобы

\begin{eqnarray}
    v^{s}_{i} & = & v_{i} + \frac{\partial E^{\text{HXC}}}{\partial n_i};
\end{eqnarray}

До этого момента не было использовано никаких приближений. Однако определение EHXC при заданной плотности является таким же требовательным, как и нахождение энергии основного состояния для заданного потенциала. Надежда заключается в том, что существуют хорошие приближения для EHXC, которые доступны с низкими численными затратами, но при этом все же позволяют получить хорошие оценки для энергии основного состояния и плотности. Здесь и в дальнейшем мы будем сравнивать три различных приближения: локальное приближение к плотности, приближение оптимизированного эффективного потенциала (так называемое приближение точного обмена) и, наконец, метод, основанный на точной диагонализации малых кластеров.

\paragraph{Моделирование квантовых точек с помощью метода конечных элементов.}

Квантовые точки полупроводников были центром внимания ученых в области конденсированного состояния в течение нескольких десятилетий. Их изменяемые свойства, опосредованные трехмерным квантовым конфинированием, привели к появлению многочисленных областей применения, включая, но не ограничиваясь, альтернативную энергетику, энергосбережение, сенсорику, квантовую оптику и фотонику, квантовую информатику и т. д. Более того, новейшие достижения в области материалов привели к разработке невероятно сложных методов производства и роста квантовых точек. Эти разработки привели к нахождению новых областей применения и созданию подкласса квантовых точек: так называемых квантовых точек с нетривиальной геометрией. Примеры таких структур варьируются от относительно простых, таких как линзообразные квантовые точки, квантовые кольца и наностержни, до более сложных, таких как нано-жабки, нано-звезды, нано-винты, тетраподы, нано-шары и т. д. Более того, эти структуры не ограничиваются одним материалом - они могут быть выращены как гетероструктуры, где одна область структуры состоит из одного материала, а другая - из другого. Хотя экспериментальные исследования сделали большой прогресс в изучении квантовых точек с нетривиальной геометрией, теоретические исследования отстают. Это обусловлено тем, что в большинстве случаев достижение электронных волновых функций и энергий для квантовых точек с нетривиальной геометрией невозможно. Некоторые исследования были проведены в рамках аппроксимации функции огибающей совместно с приближением эффективной массы для квантовых точек с сильно уплощенной и сильно вытянутой геометрией. В этих случаях геометрическое адиабатическое приближение может быть использовано для получения собственных значений и собственных функций. Однако эти случаи сильно ограничены, и даже относительно простые структуры, такие как конические квантовые точки с сопоставимым радиусом основания и высотой, не могут быть исследованы таким образом.

Таким образом, естественно, что для случаев, когда аналитическое решение невозможно, в игру вступают численные методы. Однако большинство из них, таких как методы квантовой химии, требуют значительных вычислительных мощностей, таких как кластеры или суперкомпьютеры. Более того, вычисления могут занимать чрезвычайно долгое время. Хотя, безусловно, это самые точные методы, скорость их работы делает их мало гибкими для случаев, когда исследование требует изменения геометрических параметров или изменения внешних полей.\cite{mantashyan}

\paragraph{Материалы и методы}

Самым исследуемым полупроводниковым материалом после кремния является галлий-арсенид (GaAs). У него прямая зонная структура и подходящие параметры для многих при

В текущей статье расчеты проводились с использованием коммерческого программного обеспечения под названием Wolfram Mathematica. Однако полученные результаты и методы могут быть обобщены для другого программного обеспечения, такого как MATLAB,\cite{mantashyan} COMSOL Multiphysics, и т. д. В любом расчете методом конечных элементов первый шаг - определение уравнения в частных производных. В нашем случае мы собираемся решить трехмерное уравнение Шрёдингера для одной частицы.\cite{mantashyan}

\begin{eqnarray}
    -\frac{\hbar^{2}}{2m^*} \left( \frac{\partial^2 }{\partial x^2} + \frac{\partial^2 }{\partial y^2} + \frac{\partial^2 }{\partial z^2} \right)\Psi(x,y,z) + V\cdot\Psi(x,y,z) \nonumber \\ = E\Psi(x, y, z)
    \end{eqnarray}

Здесь \(\psi(x, y, z)\) - волновая функция (собственная функция) частицы, \(V\) - потенциал удержания, а \(E\) - энергия (собственное значение) частицы. Далее нам нужно определить граничную область сетки для сеточных областей, используемых в наших расчетах, представленных на рисунке 3. Геометрические параметры, используемые для расчетов в текущей статье, представлены в таблице 1 \cite{mantashyan} .

\begin{figure}[htbp]
    \centering
    \includegraphics[width=15cm]{images/3-fig.mantashyan.png}
    \caption{\label{fig:mantashyan2} Сеточные области используются для расчетов. (а) сетка для прямоугольного КT, (б) сетка для сферического КT, (в) сетка для конического КT, (г) сетка для цилиндрического KT, (е) сетка для эллипсоида KT, (ф) сетка для эллипсоида вращения (сфероида), (г) сетка для нанорогули, (и) сетка для нанозвезды, (й) сетка для квантового кольца.\cite{mantashyan}.}
\end{figure}

Последним шагом метода конечных элементов (МКЭ) является установление граничных условий, которые обычно определяются потенциалом \( V(x, y, z) \). В большинстве наших примеров мы будем учитывать модель бесконечной потенциальной ямы, где плотность вероятности становится равной нулю за пределами структуры, то есть \( \psi(\text{outside}) = 0 \). Эта модель соответствует потенциалу вида:

\begin{eqnarray}
    V(x,y,z) = \begin{cases}
    0, & \text{внутри квантовой точки} \\
    \infty, & \text{снаружи квантовой точки}
    \end{cases}
    \end{eqnarray}
    
Это означает, что частица находится внутри квантовой точки и не может выйти за пределы структуры.

После выполнения этих шагов МКЭ разделяет область сетки на конечные элементы и решает уравнение. Во время этих процессов максимальный размер этих элементов может быть изменен, в общем, чем меньше элементы, тем выше точность. В Mathematica размер этих элементов определяется свойством, называемым MaxCellMeasure. Чтобы проверить точность решения, мы можем минимизировать энергию частицы, изменяя параметр MaxCellMeasure. Численная ошибка также зависит от геометрических параметров системы. Другой метод проверки точности метода - сравнение собственных значений, полученных с помощью МКЭ, с аналитически решаемыми случаями.\cite{mantashyan}

\subsection{Расчет энергетических уровней в сферических квантовых точках}

\paragraph{Энергия донорных примесей и оптическое поглощение в сферических секторных квантовых точках}

Квантовые точки (КТ) представляют собой кристаллические твердотельные структуры наноскопических размеров, которые можно рассматривать как квази-нулевые электронные системы, поскольку движение носителей заряда в них ограничено наличием только четко определенных значений энергии. Такой дискретный спектр привел некоторых авторов к названию этих наносистем "искусственными атомами". Они в основном состоят из полупроводниковых материалов и нашли широкое применение в различных областях технологии и науки, включая медицину. Последние достижения в области КТ описаны в работах.

Изучение полупроводниковых КТ включало различные геометрии этих структур: сферические, линзообразные и пирамидальные. Конические КТ также были исследованы, и их электронные и оптические свойства, включая эффекты донорных примесей, электрические и магнитные поля, были описаны несколькими авторами. Следует отметить экспериментальную реализацию микрокристаллов и конических гетероструктур, описанную в работах Хирума и др. и Шампа и др. Потенциальные применения этих вихрей GaAs для световых испускательных устройств являются объектом постоянных исследований, поскольку структура полупроводникового провода с использованием квантовых эффектов размера является очень важным элементом электронных и оптических устройств.

\begin{figure}[htbp]
    \centering
    \includegraphics[width=15cm]{images/4-fig.mora.png}
    \caption{\label{fig:mora1} Изобразительный обзор проекции $y = 0$ на сферическую конусообразную квантовую точку. Геометрические размеры структуры - радиус ($R$) и апикальный угол ($\theta_0$). Рассматривается примесь донора с его осевой координатой $z_i$. Сферические координаты электрона - ($r$, $\theta$, $\phi$). Потенциал конфайнмента определен как ноль внутри структуры квантовой точки и $V_0$ в других местах.\cite{mora}.}
\end{figure}

В моделировании различных свойств полупроводниковых квантовых точек с использованием метода конечных элементов (МКЭ) можно проследить до начала 1990-х годов (см., например, работу Ref.[4]). Позднее, а также в последние годы, можно упомянуть ряд исследований, посвященных структурным, электронным и оптическим свойствам, транспорту, эффектам примесей и напряжений в наносистемах в форме квантовых точек различной формы и состава. Общая среда, объединяющая методы $\mathbf{k} \cdot \mathbf{p}$ и МКЭ для расчета зонной структуры в наноструктурах, была предложена Вепреком и его коллегами.

\paragraph{Теоретическая основа}

Рассматриваемый здесь тип квантовой точки имеет коническую форму с сферической верхней крышкой, как это можно увидеть на схеме на рисунке 4. Мы предполагаем, что профиль зоны проводимости структуры включает в себя конечный потенциальный энергетический колодец ($V_0$), который практически может быть достигнут путем внедрения точечной системы в матрицу материала с большим запрещенным зазором. Без значительной потери общности, ионизированный донорный атом принимается находящимся на вертикальной оси конуса, с началом координат, совпадающим с вершиной конуса.

Разрешенные состояния электронов получаются численным методом путем решения трехмерного уравнения Шрёдингера с эффективной массой проводимости для гладко изменяющейся оболочечной волновой функции, с потенциалом конфайнмента, равным нулю внутри области конуса и $V_0=\text{const.}$ наружу. Кроме того, дифференциальное уравнение включает член кулоновского потенциала, представляющий притяжение между электроном и ионизированным центром донорной примеси. Мы учитываем зависимость эффективной массы электрона от положения, имеющую постоянные, но различные значения с обеих сторон поверхности квантовой точки. Это означает, что должны быть наложены условия соответствия типа Бен-Дэниэл-Дюка. Процесс расчета выполняется с использованием метода конечных элементов (МКЭ), реализованного в коммерческом программном обеспечении COMSOL Multiphysics [29]. Как обычно, связывающая энергия электрона с центром примеси определяется путем вычитания результата, полученного при наличии кулоновского взаимодействия, из энергии электрона в зоне проводимости без электростатического взаимодействия. На рис.5 мы представляем изобразительный обзор пространственной настройки, используемой в симуляции, рассматривая граничные условия Дирихле на достаточно большом расстоянии от активной области квантовой точки. В нашем конкретном случае материал внутри квантовой точки - GaAs, а "барьерный" регион принимается из Al0.3Ga0.7As.\cite{mora}

\begin{figure}[htbp]
    \centering
    \includegraphics[width=15cm]{images/5-fig.mora.png}
    \caption{\label{fig:mora3} Изобразительный обзор трехмерной квантовой точки внутри трехмерного кубического контейнера. Граничные условия Дирихле устанавливают, что волновая функция равна нулю на шести гранях кубической области.\cite{mora}.}
\end{figure}

Подводя итоги всех пунктов предыдущего абзаца, проблема, которую необходимо решить, соответствует электрону, заключенному в конической области радиуса $R$ и апикального угла $\theta_0$ с потенциалами конфайнмента $V(x, y, z)$, значения которых равны нулю во внутренней области конуса, $V_0$ в области вокруг квантовой точки и бесконечности во внешней области параллелепипеда (см. Рис. 5). Размеры внешнего ящика достаточно велики, чтобы можно было считать, что на носителей нет эффектов конфайнмента. Учитывая присутствие донорной примеси, расположенной на оси $z$, уравнение Шрёдингера системы выглядит следующим образом:

\begin{equation}
    -\frac{\hbar^2}{2} \nabla \cdot \left( \frac{1}{m^*(x, y, z)} \nabla \right) + V(x, y, z) - \frac{e^2}{4\pi\epsilon_0 \varepsilon_r} \frac{\sqrt{x^2 + y^2 + (z - z_i)^2}}{\sqrt{x^2 + y^2 + (z - z_i)^2}} \Psi(x, y, z) = E \Psi(x, y, z)
    \end{equation}
    
где $z_i$ представляет собой позицию примеси вдоль оси $z$, $\varepsilon_r = 13.0$ - статическая диэлектрическая константа GaAs, а $m^*(x, y, z)$ - зависящая от положения эффективная масса, значение которой составляет $0.0665 \frac{m_0}{0.092 m_0}$ в области точки/барьера (где $m_0$ - масса свободного электрона). Потенциал конфайнмента равен нулю внутри GaAs точки и 262 мэВ в окружающем материале Al$_{0.3}$Ga$_{0.7}$As.

Решение уравнения (\ref{fig:mora1}) заключается в нахождении волновых функций $\Psi(x, y, z)$ и их соответствующих энергий. Это проблема с точным аналитическим решением в тех случаях, когда потенциал конфайнмента за пределами области конуса бесконечен, а примесь расположена в его вершине. Для продолжения работы с текущей проблемой необходимо прибегнуть к численным методам, которые в нашем случае используют метод конечных элементов.

\subsection{Квантово-механический анализ на основе свойств светоизлучения квантовых точек.}

Была разработана симуляция уровней энергии квантовой точки (QD), предназначенная для воспроизведения квантово-механического аналитического метода на основе теории возмущений. Было решено уравнение Шрёдингера, описывающее пару электрон-дырку в QD, с учетом неоднородности материальных параметров ядра и оболочки. Уравнение было численно решено с использованием одночастичных базисных наборов для получения собственных состояний и энергий. Этот подход воспроизвел аналитическое решение на основе теории возмущений, хотя расчет был выполнен с использованием численного метода. Благодаря эффективности метода было надежно и легко исследовать поведение QD в зависимости от диаметра ядра и интенсивности внешнего электрического поля. 9,2 нм квантовая точка CdSe/ZnS с ядром диаметром 4,2 нм и оболочкой толщиной 2,5 нм излучала зеленый свет с длиной волны 530 нм, согласно анализу влияния диаметра ядра на уровни энергии. Было обнаружено красное смещение на 4 нм при интенсивности внешнего электрического поля 5,4 × 105 В/см при исследовании влияния внешнего электрического поля на уровни энергии. Эти значения хорошо соответствуют ранее сообщенным экспериментальным результатам. Кроме уровней энергии и длин волн световой эмиссии, были получены пространственные распределения волновых функций. Этот метод анализа широко применим для изучения характеристик QD с изменяющейся структурой и составом материалов и должен способствовать развитию высокопроизводительных технологий QD.

Квантовая точка (QD) является отличным оптоэлектронным материалом среди различных наночастиц. Ее спектры световой эмиссии острые и легко контролируются путем изменения диаметра. Кроме того, стабильные слои светоизлучения QD (EML) могут быть изготовлены благодаря неорганической природе QD. Учитывая эти преимущества, продолжаются усилия по разработке высокопроизводительных QD и светодиодов, использующих QD EML, с целью замены органических светодиодов (OLED) на светодиоды с квантовыми точками (QLED). Однако исследования по QD и QLED находятся на начальном этапе; поэтому их срок службы и эффективность должны быть дальше улучшены перед коммерческим применением QLED. Требуется детальный анализ энергетических уровней QD, поскольку они определяют оптоэлектронные характеристики. Квантовые эффекты конфайнмента модулируют энергетические уровни QD и свойства люминесценции. Кроме того, квантово-ограниченный эффект Штарка (QCSE), возникающий из-за напряженности напряжения, влияет на электролюминесцентные свойства QD. Несмотря на их важность, экспериментально сложно анализировать квантовые эффекты. Таким образом, требуются теоретические и/или численные анализы. Квантовые явления, связанные с QD, могут быть описаны уравнением Шрёдингера, которое решается с учетом профиля потенциала, материальных свойств (таких как эффективные массы и диэлектрические константы) и кулоновских взаимодействий в QD.

Весь гамильтониан можно решить с использованием метода конечных разностей (FDM) или метода конечных элементов (FEM); эти методы твердо установлены в терминах их способности предоставлять сложные численные результаты. Однако аналитические решения более подходят для интерпретации квантово-механического поведения. Как практический способ получения аналитических решений используется метод возмущений. Невозмущенный гамильтониан состоит из одночастичных терминов гамильтониана, представляющих электрон и дырку. Профили электрического потенциала, эффективной массы и диэлектрической константы, а также кулоновского взаимодействия между электроном и дыркой, учитываются в терминах возмущения. Несмотря на полезность аналитических решений на основе этого метода возмущений, требуется значительное усилие для получения аналитических решений в различных условиях, таких как диаметр QD, интенсивность электрического поля и свойства материала. Полуаналитический метод может преодолеть это, поскольку он имитирует аналитический метод с использованием численного метода. В предыдущей работе уравнение Шрёдингера для QD было решено на основе одночастичных уравнений Шрёдингера, используя термины возмущения, соответствующие энергии кулоновского взаимодействия между электроном и дыркой и потенциальной энергии, связанной с профилями зон энергии. Однако профили эффективной массы и диэлектрической константы были игнорированы для упрощения анализа, что могло вызвать лишние ошибки. В этой работе полуаналитический метод значительно улучшен для более точного анализа. Профили эффективной массы и диэлектрической константы включены в термины возмущения, и учитываются образные заряды при расчете энергии кулона. С использованием этого подхода получается эффективный метод для понимания квантовых явлений в QD. Зависимость длины световой волны, излучаемой из QD, и формы волновых функций в QD от диаметра QD и интенсивности внешнего электрического поля была исследована с использованием предложенного подхода. Для проверки эффективности предложенного метода свойства ядра/оболочки CdSe/ZnS QDs были проанализированы с использованием этого метода. CdSe/ZnS стал основной структурой QD из-за его легкости синтеза и высокой производительности. Следовательно, надежные сравнительные экспериментальные данные по CdSe/ZnS QDs могут быть получены относительно легко. Это делает CdSe/ZnS QDs подходящим тестовым средством для этой работы.

\paragraph{Результаты и обсуждение}

Эффекты квантового ограничения были исследованы с использованием описанного выше метода анализа. Были проведены симуляции квантовых точек с ядро/оболочка структурой CdSe/ZnS. На рисунке показана структура квантовой точки и материальные параметры. Диаметр ядра, $d_{\text{core}}$, был оценен с использованием соотношения $d_{\text{core}}/l = \sqrt{6/\pi}$, где $l$ - одномерная длина ядра; это соотношение было найдено путем сравнения объемов сферической реальной квантовой точки и модельной квадратной квантовой точки, использованной в данном анализе. Значение энергии запрещенной зоны ядра квантовой точки $E_g$ составляет 1.74 эВ. На границе между ядром и оболочкой присутствуют барьеры энергии 1.27 эВ для зоны проводимости и 0.6 эВ для зоны валентности. $m_{e,c}^*, m_{h,c}^*, m_{e,s}^*,$ и $m_{h,s}^*$ равны 0.13-, 0.45-, 0.34- и 0.23-кратной массе покоя электрона, соответственно. Относительная диэлектрическая проницаемость составляет 6.36 в ядре и 5.71 в оболочке. Толщина оболочки, $t_{\text{shell}}$, была зафиксирована на уровне 2.5 нм. Зависимость энергии запрещенной зоны и длины волны излучения от диаметра квантовой точки показана на рисунке . С увеличением диаметра ядра энергия запрещенной зоны уменьшается, а длина волны увеличивается. Эти явления объясняются эффектами квантового ограничения. Диаметр ядра квантовой точки CdSe/ZnS, излучающей зеленый свет с длиной волны 530 нм, составляет 4.2 нм; следовательно, общий диаметр квантовых точек составляет 9.2 нм, что соответствует предыдущим экспериментальным данным.\cite{lee}

\begin{figure}[htbp]
    \centering
    \includegraphics[width=18cm]{images/6-fig.lee.png}
    \caption{\label{fig:lee} (a) Энергия запрещенной зоны и длина волны в зависимости от интенсивности внешнего электрического поля,(b) Плотности вероятности в зависимости от одномерного расстояния между электроном и дыркой, и распределения волновых функций основного состояния в зависимости от положения электрона и дырки (c) без и (d) с внешним электрическим полем. В легенде показаны значения интенсивности внешнего электрического поля.\cite{lee}.}
\end{figure}

Зависимость энергии запрещенной зоны и длины волны излучения от интенсивности внешнего электрического поля показана на рисунке \ref{fig:lee}. Диаметр ядра и толщина оболочки составляли 4.2 и 2.5 нм соответственно, так что диаметр квантовой точки составлял 9.2 нм. Внешнее электрическое поле прикладывалось вдоль отрицательного направления. С увеличением интенсивности внешнего электрического поля энергия запрещенной зоны уменьшалась, а длина волны излучения увеличивалась. Это поведение согласуется с предыдущими экспериментальными результатами и объясняется QCSE. Электрическое поле 4.7 × $10^7$ В/см привело к красному сдвигу длины волны излучения на 4 нм. Красный сдвиг около 4 нм был легко наблюдаем с использованием QLED с EML из CdSe/ZnS QD. Электрическое поле 4.7 × $10^7$ В/см соответствует падению напряжения 0.43 В на квантовую точку диаметром 9.2 нм. Таким образом, для хорошо изготовленного QLED большая часть смещения напряжения применяется к слоям транспорта заряда, и только небольшая часть смещения напряжения применяется к квантовой точке. Одномерные плотности вероятности основных состояний в зависимости от расстояния между электроном и дыркой с внешним электрическим полем и без него показаны на рисунке \ref{fig:lee}. Ожидаемое значение расстояния между электроном и дыркой увеличивается при применении электрического поля, как показано на рисунке. Более подробные поведения носителей в основных состояниях показаны на рисунках \ref{fig:lee}c и \ref{fig:lee}d. Положения электрона и дырки были почти симметричны без электрического поля (рисунок \ref{fig:lee}c); однако симметрия нарушалась при применении электрического поля (рисунок \ref{fig:lee}d).

\paragraph{Заключение}

Симуляция уровня энергии QD была разработана для воспроизведения квантово-механического аналитического решения на основе теории возмущений. Симуляция была проведена с использованием численного метода. С помощью этого подхода поведение электрона и дырки в QD могло быть эффективно проанализировано на основе уравнения Шрёдингера с учетом структуры ядра/оболочки и материальных свойств. Поскольку этот подход имитирует аналитический метод возмущений, результаты эквивалентны аналитическим решениям на основе метода возмущений. Благодаря эффективности предложенного метода можно было надежно и легко проанализировать поведение QD в зависимости от диаметра ядра и интенсивности внешнего электрического поля. Изучалась зависимость длины волны излучения света от диаметра ядра; квантовая точка CdSe/ZnS с диаметром ядра 9.2 нм и диаметром ядра 4.2 нм и толщиной оболочки 2.5 нм излучала зеленый свет с длиной волны 530 нм. Был изучен эффект красного сдвига, вызванный электрическим полем, и привел к сдвигу $\sim$4 нм при интенсивности электрического поля $5.4 \times 10^5$ В/см. Диаметр QD, величина красного сдвига и интенсивность электрического поля для красного сдвига хорошо согласуются с ранее опубликованными экспериментальными результатами. В дополнение к энергетическим уровням и длинам волн излучения света, были определены пространственные распределения волновых функций; эта информация помогает объяснить поведение носителей в QD, что может привести к улучшению структур QD и QLED. Таким образом, этот метод анализа широко применим для изучения характеристик QD для различных структур и составов материалов. Несмотря на успех этой работы, желательно провести дополнительные исследования для дальнейшего улучшения метода анализа, такие как расчет вероятностей переходов между квантовыми зоны. 

\subsection{Синтез квантовых точек и перспективы будущих применений}

Российские исследователи обнаружили квантование уровней в квантовых точках (КТ) в 1981 году. Кроме того, был получен синий сдвиг в оптическом спектре для нано-CuCl в кремниевом стекле. Римляне и греки использовали квантовые точки сульфида свинца(II) (PbS) в качестве красителя для волос более 2 000 лет, приготавливая эти материалы из натуральных компонентов, таких как оксид свинца, гидроксид кальция и вода. Кроме того, 4 000 лет назад, Древний Египет использовал нано-PbS для различных цветов красительных формул для волос.

В течение $100$ лет одним из старейших методов контроля цвета стекла является контроль размера квантовых точек в кремниевых стеклах. В последние несколько десятилетий кадмий селенид ($\text{CdSe}$) и кадмий сульфид ($\text{CdS}$) были объединены в кремниевые стекла, чтобы получить разные цвета, варьирующиеся от красного до желтого. Впервые рентгеновская дифракция была использована в $1932$ году для определения того, что осаждение $\text{CdSe}$ и $\text{CdS}$ вызывает появление красно-желтого цвета. В период с $1982$ по $1993$ год квантовый размерный эффект играл ключевую роль в контроле цвета стекла путем изменения размера квантовых точек несколькими различными методами синтеза, которые были разработаны в течение этого периода. Таким образом, в $1998$ году компании начали продавать продукцию на основе квантовых точек, например, компания Quantum Dot Corporation в США, которая привлекла более $37.5$ млн инвестиций венчурного капитала.

За последние два десятилетия квантовые точки (QD) претерпели мутацию в своем разнообразии, что привело к точному контролю наноразмеров QD и возможности использования их в передовых приложениях. Одно из таких приложений революционизирует мир дисплеев и здравоохранения, особенно в лечении серьезных заболеваний, таких как лейкоз, различные виды рака и многие биомедицинские приложения, очистка сточных вод и солнечные элементы QDS.

\paragraph{Размер квантовых точек (QD)}

Каждые 10 нм содержат около 3 миллионов QD, когда они выстроены вместе или помещены в ширину человеческого пальца. Из-за маленького размера QD электроны QD ограничены в квантовом ящике (маленьком пространстве), также когда их радиусы меньше боровского радиуса экситона, что означает, что разделение между электроном и дыркой в паре электрон-дырка следует за большим разделением уровней энергии. Это приводит к необходимости большего количества энергии для перехода в возбужденное состояние, и увеличивается энергия при возвращении к энергии покоя.

Обычно, по мере уменьшения размера QD, разница в энергии между самой высокой валентной зоной и самой низкой зоной проводимости увеличивается. Следовательно, возбуждение точки требует больше энергии, которая высвобождается, когда их кристалл возвращается к своему основному состоянию, из-за сдвига цвета от более длинной волны (красный цвет) к более короткой волне (синий цвет) в испускаемом свете (рисунок 7).\cite{mohamed}

\begin{figure}[htbp]
    \centering
    \includegraphics[width=13cm]{images/7-fig.mohamed.png}
    \caption{\label{fig:mohamed} Разделение энергетических уровней в QD вследствие квантового эффекта ограничения и увеличение запрещенной зоны полупроводника с уменьшением размера нанокристалла.\cite{mohamed}.}
\end{figure}

Электронное возбуждение для полупроводникового нанокристалла зависит от местоположения пары электрон-дырка, которое обычно деокализовано на большем расстоянии, чем постоянная решетки.

\begin{figure}[htbp]
    \centering
    \includegraphics[width=18cm]{images/8-fig.mohamed.png}
    \caption{\label{fig:mohamed1} Поведение настройки квантовых точек.\cite{mohamed}.}
\end{figure}

\paragraph{Эффект размера квантовых точек.}

Эффект изменения цвета, связанный с размером квантовых точек (QDE), возникает, когда диаметр полупроводникового нанокристалла приближается к диаметру боровского экситона, что вызывает начальное изменение электронных свойств QD. CdS, один из известных квантовых точек, был исследован 30 лет назад; QD образуются, когда их нанодиаметр находится около или ниже диаметра экситона \( \leq 6 \) нм (\(3,000-4,000\) атомов).

Обычно большой процент атомов в QD находится в режиме малого размера, находясь на или рядом с поверхностью. Квантовая точка кадмия сульфида размером $5$ нм имеет $15\%$ своих атомов на поверхности. Существование большого интерфейса между нанокристаллом и окружающей средой влияет на их свойства. Нанокристаллы имеют несовершенную поверхность и электронные и ловушки, наблюдаемые при оптическом возбуждении. Таким образом, наличие этих пойманных электронов и дырок привело к изменениям оптических свойств нанокристаллов.

\paragraph{Феномен изменения цвета}

Квантовые точки (QD) могут излучать любой цвет света из одного и того же полупроводникового нанокристалла, просто в зависимости от изменений его размера. Это относится к высокому уровню контроля, достижимому над размером QD, который можно регулировать во время процессов синтеза для излучения любого цвета света. Следовательно, более крупные точки излучают на более длинных волнах, например, красный, в то время как более мелкие точки излучают на более коротких волнах, например, зеленый. Тюнинг QD (длина волны излучаемого света) имеет тот же характер, что и струна гитары, потому что более высокая высота тона производится, когда струна гитары короткая, а более низкая - когда она удлиняется (см. Рисунок 8).

В то же время ширина запрещенной зоны изменяется в зависимости от размера, причем ширина запрещенной зоны - это энергия, необходимая для перевода электрона из валентной зоны в зону проводимости, и когда она находится в диапазоне видимого спектра длин волн, это приводит к изменениям в излучаемом цвете. Кроме того, магнитная память (коэрцитивная сила), необходимая для обращения внутреннего магнитного поля внутри квантовой точки, зависит от ее размера.

\paragraph{Квантовое ограничение}

В нанокристаллических полупроводниках размером вдвое больше радиуса Бора экситона возникает квантовое ограничение, когда экситоны сжимаются. Самое существенное последствие этого эффекта - зависимость ширины запрещенной зоны от размера, которая настраивается на определенную энергию в зависимости от степени ограничения и размерности нанокристалла.

Разделение энергетических уровней влияет на стационарные волновые функции для различных форм QD (прямоугольной и пирамидальной). Энергетические состояния в точках формы прямоугольника сохраняют орбитальную симметрию, являясь более характерными для s-типа и p-типа. В то время как в точках формы пирамиды волновые функции смешиваются из-за эффекта асимметричного ограничения (см. Рисунок 9).

\begin{figure}[htbp]
    \centering
    \includegraphics[width=15cm]{images/9-fig.mohamed.png}
    \caption{\label{fig:mohamed7} 3D ограниченные волновые функции электрона в квантовой точке.\cite{mohamed}.}
\end{figure}

\paragraph{Структура квантовой точки.}

QD обычно состоят из 200–3,000,000 атомов, но имеют только 100 свободных электронов или меньше [18]. Многие структуры могут быть различимы по их ограничению электронов. Первые эксперименты с QD были проведены с плоскими структурами, где их электростатическое ограничение приводит к размерам около 100 нм, а структурное ограничение для вертикальных QD составляет около 10 нм. Хотя самосборочные структуры QD имеют пирамидальную или линзообразную форму \( \leq 10 \) нм, их электростатическое ограничение позволяет считать, что пирамидальные QD более перспективны для лазерных применений, чем плоские и вертикальные структуры.

\begin{figure}[htbp]
    \centering
    \includegraphics[width=15cm]{images/10-fig.mohamed.png}
    \caption{\label{fig:mohamed8} Структуры с различными направлениями ограничения.\cite{mohamed}.}
\end{figure}

В общем, эффекты квантования в нанокристаллических полупроводниковых структурах имеют уменьшенный размер, зависящий от направления конфайнинга носителей заряда в различных измерениях (Рисунок 10).

\paragraph{Ядерные квантовые точки}

Квантовые точки могут быть однокомпонентными материалами с равномерными внутренними составами, такими как селениды или сульфиды (халькогениды). Электролюминесцентные и оптические свойства ядерных квантовых точек могут быть настраиваемы при любом простом изменении размера квантовой точки.

\begin{figure}[htbp]
    \centering
    \includegraphics[width=15cm]{images/11-fig.mohamed.png}
    \caption{\label{fig:mohamed9} Структура ядра квантовой точки.\cite{mohamed}.}
\end{figure}

\paragraph{Ядерно-оболочечные квантовые точки (CSQD)}

Квантовые точки с небольшими областями одного материала, встроенными в другой материал с запрещенной зоной, известны как CSQD. Квантовые точки с \( \text{CdSe} \) в ядре и \( \text{ZnS} \) в оболочке доступны с высоким квантовым выходом \(\geq 80\%\). Оболочки квантовых точек улучшают квантовый выход и делают их более подходящими для условий их синтеза для различных приложений (рисунок \ref{fig:mohamed9}).

CSQDs используются для повышения эффективности и яркости полупроводниковых нанокристаллов, и к ним прирастают оболочки другого полупроводникового материала с более высокой запрещенной зоной. Следовательно, электролюминесцентные свойства QD возникают из-за распада экситона (рекомбинации пар электрон-дырка) через радиационные процессы. Также это может происходить через нерадиационные процессы, приводящие к снижению квантовой отдачи флуоресценции.

\section*{Проблемы исследования}

\subsection*{Расчет положения уровней энергии в квантовых точках}

Для различных форм и размеров квантовых точек требуется расчет положения энергетических уровней, что позволит оценить их влияние на электронные свойства системы. Используемые физические параметры включают размеры квантовых точек, их форму, а также потенциальную энергию внешнего воздействия.

\subsection*{Анализ уровней энергии для кубических квантовых точек}

Для кубических квантовых точек требуется провести детальный анализ энергетических уровней, учитывая их зависимость от геометрических параметров. Это позволит оптимизировать структуры для конкретных приложений, таких как транзисторы и светодиоды.

\subsection*{Анализ уровней энергии для сферических квантовых точек}

Для сферических квантовых точек также необходимо исследовать энергетические уровни, учитывая их геометрические параметры. Это позволит сравнить их с результатами для кубических квантовых точек и определить особенности электронных свойств в различных геометриях.

\subsection*{Расчет волновых функций для кубических квантовых точек}

Для кубических квантовых точек требуется расчет волновых функций электронов, что позволит оценить их пространственное распределение и взаимодействие с окружающими структурами. В расчетах учитываются геометрические параметры и потенциальная энергия.

\subsection*{Расчет волновых функций для сферических квантовых точек}

Аналогично для сферических квантовых точек требуется провести расчеты волновых функций, учитывая их геометрические параметры и взаимодействие с окружающей средой. Это позволит определить особенности электронных состояний в сферических структурах.

\subsection{Раздел 2}

% ============================================
% ГЛАВА 2
% ============================================
\pagebreak
\section{Теория и основные уравнения}

\subsection{Раздел 1}

Ненумерованная формула:

\begin{equation}
    \begin{pmatrix} \dot{\varphi}\\ \dot{\theta} \\ \dot{\psi} \end{pmatrix}
    = \begin{pmatrix}
        \cos(\theta)\cos(\psi) & -\sin(\psi) & 0 \\
        \cos(\theta)\sin(\psi) & \cos(\psi)  & 0 \\
        -\sin(\theta)         & 0         &  1
    \end{pmatrix}^{-1}
    \begin{pmatrix} \omega_x\\ \omega_y \\ \omega_z \end{pmatrix}. \nonumber
\end{equation}


\subsection{Раздел 2}

Нумерованные формулы:

\begin{equation}
\label{eq:1}
    \dot{\theta}=\frac{P-p_{1}\cos\left(\varphi_{1}-\theta\right)-p_{2}\cos\left(\varphi_{2}-\theta\right)}{\mu+\sin^{2}\left(\varphi_{1}-\theta\right)+\sin^{2}\left(\varphi_{2}-\theta\right)}
\end{equation}

\begin{equation}
    \dot{\varphi}_{1}=p_{1}-\dot{\theta}\cos(\phi_{1}-\theta)
\end{equation}

\begin{equation}
    \dot{\varphi}_{2}=p_{2}-\dot{\theta}\cos(\phi_{2}-\theta)
\end{equation}

Тест ссылки на формулу (\ref{eq:1}).

% ============================================
% ГЛАВА 3
% ============================================
\pagebreak
\section{Численные методы и алгоритмы}

\subsection{Раздел 1}

\subsection{Раздел 2}

% ============================================
% ГЛАВА 4
% ============================================
\pagebreak
\section{Программная реализация}

\begin{lstlisting}[language=rust,caption={Программная реализация метода Рунге-Кутты},label={listing-1}]
    // From the pendulum program
    fn runge_kutta(
        vars: &MyVec,
        pars: &Vec<f64>,
        rhs: &dyn Fn(&MyVec, &Vec<f64>) -> MyVec,
        dt: f64,
    ) -> MyVec {
        let rk_1 = rhs(vars, pars);
        let rk_2 = rhs(&vars.add(&rk_1.scale(dt / 2.0)), pars);
        let rk_3 = rhs(&vars.add(&rk_2.scale(dt / 2.0)), pars);
        let rk_4 = rhs(&vars.add(&rk_3.scale(dt)), pars);
    
        let vars_new = vars
            .add(&rk_1.scale(dt / 6.0))
            .add(&rk_2.scale(dt / 3.0))
            .add(&rk_3.scale(dt / 3.0))
            .add(&rk_4.scale(dt / 6.0));
        vars_new
    }
    \end{lstlisting}
    
    \begin{lstlisting}[language=C++,caption={Подпрограмма случайного блуждания на плоскости},label={listing-2}]
    std::random_device rd;
    std::mt19937 mt(rd());
    std::uniform_int_distribution<long> dist(1, 4);
    std::vector<long> xn(n0, 0);
    std::vector<long> yn(n0, 0);
    for (long jt = 0; jt < M; jt++)
    {
        for (long jn = 0; jn < n0; jn++)
        {
            switch (dist(mt))
            {
            case 1:
                xn[jn] ++;
                break;
            case 2:
                xn[jn] --;
                break;
            case 3:
                yn[jn] ++;
                break;
            case 4:
                yn[jn] --;
                break;
            }
        }
    }
    \end{lstlisting}

% ============================================
% ГЛАВА 5
% ============================================
\pagebreak
\section{Результаты и обсуждение}

Ниже тестируется очень большая таблица на несколько страниц

\begin{center}
    \begin{longtable}{|p{2cm}|p{3cm}|p{7cm}|p{3cm}|}
    \caption{Заголовок таблицы}\\
    \hline
    1 & 2 & 3 & 4\\ 
    \hline 
    2 & 2 & 3 & 4\\
    \hline
    3 & 2 & 3 & 4\\
    \hline
    4 & 2 & 3 & 4\\
    \hline
    5 & 2 & 3 & 4\\
    \hline
    6 & 2 & 3 & 4\\
    \hline
    7 & 2 & 3 & 4\\
    \hline
    8 & 2 & 3 & 4\\
    \hline
    9 & 2 & 3 & 4\\
    \hline
    10 & 2 & 3 & 4\\
    \hline
    1 & 2 & 3 & 4\\ 
    \hline 
    2 & 2 & 3 & 4\\
    \hline
    3 & 2 & 3 & 4\\
    \hline
    4 & 2 & 3 & 4\\
    \hline
    5 & 2 & 3 & 4\\
    \hline
    6 & 2 & 3 & 4\\
    \hline
    7 & 2 & 3 & 4\\
    \hline
    8 & 2 & 3 & 4\\
    \hline
    9 & 2 & 3 & 4\\
    \hline
    10 & 2 & 3 & 4\\
    \hline
    1 & 2 & 3 & 4\\ 
    \hline 
    2 & 2 & 3 & 4\\
    \hline
    3 & 2 & 3 & 4\\
    \hline
    4 & 2 & 3 & 4\\
    \hline
    5 & 2 & 3 & 4\\
    \hline
    6 & 2 & 3 & 4\\
    \hline
    7 & 2 & 3 & 4\\
    \hline
    8 & 2 & 3 & 4\\
    \hline
    9 & 2 & 3 & 4\\
    \hline
    10 & 2 & 3 & 4\\
    \hline
    
    
    \end{longtable}
\end{center}


А также тестируется счетчик таблиц, жирные и двойные линии.

\begin{center}
    \begin{longtable}{|p{2cm}||p{3cm}|p{7cm}|p{3cm}|}
    \caption{Заголовок таблицы номер 2}\\
    \hline
    1 & 2 & 3 & 4\\ 
    \hline
    2 & 2 & 3 & 4\\
    \hline
    3 & 2 & очень жирная ячейка \par с переносом & 4\\
    \hline
    4 & 2 & 3 & 4\\
    \hline
    5 & 2 & 3 & 4\\
    \hline
    6 & 2 & 3 & 4\\
    \hline
    7 & 2 & 3 & 4\\
    \hline
    8 & 2 & 3 & 4\\
    \hline
    9 & 2 & 3 & 4\\
    \hline
    10 & 2 & 3 & 4\\
    \hline
    
    
    \end{longtable}
\end{center}

Ссылаемся на Листинг \ref{listing-1} здесь.

% ============================================
%  ВЫВОДЫ И ЗАКЛЮЧЕНИЕ
% ============================================
\pagebreak
\specialsection{Выводы}
Структура файлов, которые можно редактировать:

\begin{itemize}
    \item \verb|main.tex| --- содержит основной текст;
    \item \verb|titlepage.tex| --- содержит титульный лист;
    \item \verb|literature.bib| --- содержит источники для списка литературы;
    \item \verb|code_highlight.tex| --- форматирование листингов (фрагментов кода).
\end{itemize}

Файл \verb|diploma.sty| очень важный, его трогать и особенно удалять не надо, там задаются различные стили документа.

\specialsection{Заключение}
Нужны ли отдельно и выводы, и заключение --- я не знаю. Разберёмся.

Список литературы ниже оформлен не по ГОСТу, но это легко исправить. Главное, что он организован, и можно ссылаться на каждый пункт по фамилии первого автора.

\textbf{Внимание!} 

Список литературы находится в отдельном файле \verb|literature.bib|, в который можно добавлять новые источники в любом порядке. Они будут сами располагаться как нужно, в порядке упоминания в тексте.

Если какой-то источник не процитирован в тексте, он в список литературы добавлен не будет.

Поэтому один и тот же файл с источниками можно использовать для нескольких документов.

\printbibliography

\end{document}